%%---------------------------------------------
% Präambel
\documentclass[12 pt,german,a4paper]{article}

\usepackage[utf8]{inputenc}    
\usepackage[T1]{fontenc}
\usepackage[german]{babel} 

\renewcommand*\rmdefault{lmss}
\usepackage{setspace} 

\usepackage{color, colortbl}
\definecolor{grau}{rgb}{.9,.9,.9} 

\usepackage{bbding}
\usepackage{dingbat}

\usepackage[table]{xcolor} 
\usepackage{array}
\usepackage{multirow}
\usepackage{multicol}
\usepackage{graphicx}
\usepackage{fp}
\usepackage{textpos}
\usepackage{dashrule}
\usepackage{tikz}
\usepackage{paralist} 


\begin{document}
\pagestyle{empty}

\section*{Inhalt des Fragebogens}

\noindent
Guten Tag, \\
mein Name ist Stefan Munnes und ich führe im Rahmen meiner Masterarbeit eine Befragung zu allgemeinen gesellschafts-politischen Einstellungen und der Beziehung zum Judentum durch. Mit Ihrer Teilnahme ermöglichen Sie mir wichtige empirische Daten für mein Forschungsvorhaben zu erheben. Dafür möchte ich mich bei Ihnen bedanken. \\
Fragen und Anregungen zu dem vorliegenden Fragebogen können Sie mir gerne zukommen lassen: munnes$@$uni-potsdam.de \newline \newline
Im Folgenden finden Sie Hinweise zum Fragebogen und eine Datenschutzerklärung, die Sie behalten können.



\section*{Hinweise zum Ausfüllen}
\begin{itemize}
\item Die Beantwortung der Fragen ist freiwillig und dauert ca. 10 Minuten.
\item Bei den meisten Fragen kommt es auf Ihre persönliche Einstellung an.
\item Bitte kreuzen Sie pro Frage das Kästchen mit der für Sie zutreffenden Antwort an. \includegraphics[scale=.4]{kaestchen_kreuz.png}
\item Falls Sie sich umentscheiden wollen, füllen Sie das falsch angekreuzte Kästchen ganz aus. \includegraphics[scale=.4]{kaestchen_voll.png}
\end{itemize}


\section*{Datenschutzerklärung}
Hiermit erkläre ich, dass alle erhobenen Daten stehts vertraulich behandelt und weiterverarbeitet werden. Es werden keine unnötigen personenbezogenen Daten erfasst. Die Ergebnisse werden ausschließlich in Gruppen zusammengefasst und dargestellt. Das bedeutet: Niemand kann in der Auswertung und späteren Darstellung erkennen, von welcher Person die Angaben gemacht worden sind.
\\
\vspace{1cm}
\\
\begin{tabular}{p{8cm} m{5cm}}
Potsdam, 08.04.2018 & \includegraphics[scale=1]{unterschrift.png}
\end{tabular} 




\end{document}