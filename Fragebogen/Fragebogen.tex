%%---------------------------------------------
% Präambel
\documentclass[12 pt,german,a4paper]{article}

\usepackage[utf8]{inputenc}    
\usepackage[T1]{fontenc}
\usepackage[german]{babel} 

\renewcommand*\rmdefault{lmss}
\usepackage{setspace} 

\usepackage{color, colortbl}
\definecolor{grau}{rgb}{.9,.9,.9} 

\usepackage{bbding}
\usepackage{dingbat}

\usepackage[table]{xcolor} 
\usepackage{array}
\usepackage{multirow}
\usepackage{multicol}
\usepackage{graphicx}
\usepackage{fp}
\usepackage{textpos}
\usepackage{dashrule}
\usepackage{tikz}
\usepackage{paralist} 

% Für Fragebogen
%---------------------------------------------
% Bienvenue Stylepaket
\usepackage{automultiplechoice}    
% LMES Stylepaket
\usepackage{zusatz}

%---------------------------------------------
% AMC anpassen

\AMCtext{message}{\footnotesize \color{gray}}
\AMCtext{draft}{}

% Vertikaler Abstand zwischen den einzelnen Fragen
\def\AMCbeginQuestion#1#2{\vspace{0cm}} 

%Größe Antwortboxen
%\AMCboxDimensions{size=1.7ex}

%---------------------------------------------
% Beginn Dokument

\begin{document}
\setlength{\parindent}{0pt}

% Anzahl individualsierter Fragebögen
\onecopy{1}{
	 
\stepcounter{mysection}

%---------------------------------------------
% Fragebogen
\section*{Allgemeine Angaben}
Zu Beginn beantworten Sie bitte einige allgemeine Fragen. \\

\fr{Was ist Ihr Geschlecht?}
\vspace{-0.2cm}
\begin{questionmult}{sex}\scoring{v=-1}
	\begin{choices}[o]
		\correctchoice{männlich}\scoring{b=0} 
		\correctchoice{weiblich}\scoring{b=1}
		\correctchoice{weiteres}\scoring{b=3}
	\end{choices}
\end{questionmult} 

\vspace{0.3cm}

\fr{Wie alt sind Sie zum jetzigen Zeitpunkt?}
\vspace{-0.2cm}
\begin{questionmult}{age}\scoring{v=-1}
	\begin{choices}[o]
		\correctchoice{16-20 Jahre}\scoring{b=1}
		\correctchoice{21-25 Jahre}\scoring{b=2}
		\correctchoice{26-30 Jahre}\scoring{b=3}
		\correctchoice{31 Jahre und älter}\scoring{b=4}
	\end{choices}
\end{questionmult} 

\vspace{0.3cm}

\fr{Wie beurteilen Sie ganz allgemein die heutige wirtschaftliche Lage in Deutschland?}
\vspace{-0.2cm}
\begin{questionmult}{lage}\scoring{v=-1}
	\begin{choices}[o]
		\correctchoice{sehr gut}\scoring{b=1}
		\correctchoice{gut}\scoring{b=2}
		\correctchoice{teils gut, teils schlecht}\scoring{b=3}
		\correctchoice{schlecht}\scoring{b=4}
		\correctchoice{sehr schlecht}\scoring{b=5}
	\end{choices}
\end{questionmult} 

\vspace{0.3cm}

\fr{Wie stark interessieren Sie sich für Politik?}
\vspace{-0.2cm}
\begin{questionmult}{polint}\scoring{v=-1}
	\begin{choices}[o]
		\correctchoice{sehr stark}\scoring{b=1} 
		\correctchoice{stark}\scoring{b=2}
		\correctchoice{mittel}\scoring{b=3}
		\correctchoice{wenig}\scoring{b=4}
		\correctchoice{gar nicht}\scoring{b=5}
	\end{choices}
\end{questionmult} 

\vspace{0.3cm}	


\fr{Viele Leute verwenden die Begriffe "`links"' und "`rechts"', wenn es darum geht unterschiedliche politische Einstellungen zu kennzeichnen.} \\

\headerfive{7.8cm}{links}{}{}{}{rechts} \\
\vspace{-.5cm}
	\itemsfive{v5a}{8cm}{Wenn Sie an Ihre eigenen politischen Ansichten denken, wo würden Sie sich auf dieser Skala einstufen?}{1}{2}{3}{4}{5} \\  	




\pagebreak

\section*{Gesellschaftspolitische Einstellung}
In diesem Abschnitt finden Sie einige allgemeine Aussagen über gesellschaftliche und politische Einstellungen. Bitte geben Sie an, inwieweit Sie diesen zustimmen.
\\

\fr{Wir hätten heute weniger Probleme wenn sich der Mensch nicht so sehr von seiner Natur entfernt hätte.}
\vspace{-0.2cm}
\begin{questionmult}{latent01}\scoring{v=-1}
	\begin{choices}[o]
		\correctchoice{stimme voll zu}\scoring{b=1} 
		\correctchoice{stimme eher zu}\scoring{b=2}
		\correctchoice{stimme eher nicht zu}\scoring{b=3}
		\correctchoice{stimme gar nicht zu}\scoring{b=4}
		\vspace{-0.05cm}
		\\ . . . . . . . . . . . 
		\vspace{-0.05cm}
		\correctchoice{keine Angabe}\scoring{b=88}
	\end{choices}
\end{questionmult} 

\vspace{0.3cm}

\fr{Mir fällt es schwer zu glauben, dass eine kleine Minderheit in der Lage ist uns alle zu lenken.}
\vspace{-0.2cm}
\begin{questionmult}{latent02}\scoring{v=-1}
	\begin{choices}[o]
		\correctchoice{stimme voll zu}\scoring{b=1} 
		\correctchoice{stimme eher zu}\scoring{b=2}
		\correctchoice{stimme eher nicht zu}\scoring{b=3}
		\correctchoice{stimme gar nicht zu}\scoring{b=4}
		\vspace{-0.05cm}
		\\ . . . . . . . . . . . 
		\vspace{-0.05cm}
		\correctchoice{keine Angabe}\scoring{b=88}
	\end{choices}
\end{questionmult} 

\vspace{0.3cm}

\fr{Wir müssen endlich aufhören uns gegeneinander aufhetzen zu lassen.}
\vspace{-0.2cm}
\begin{questionmult}{latent03}\scoring{v=-1}
	\begin{choices}[o]
		\correctchoice{stimme voll zu}\scoring{b=1} 
		\correctchoice{stimme eher zu}\scoring{b=2}
		\correctchoice{stimme eher nicht zu}\scoring{b=3}
		\correctchoice{stimme gar nicht zu}\scoring{b=4}
		\vspace{-0.05cm}
		\\ . . . . . . . . . . . 
		\vspace{-0.05cm}
		\correctchoice{keine Angabe}\scoring{b=88}
	\end{choices}
\end{questionmult} 

\vspace{0.3cm}

\fr{Unter uns gibt es Mächte, deren einziges Ziel die Zerstörung unserer Gemeinschaft ist.}
\vspace{-0.2cm}
\begin{questionmult}{latent04}\scoring{v=-1}
	\begin{choices}[o]
		\correctchoice{stimme voll zu}\scoring{b=1} 
		\correctchoice{stimme eher zu}\scoring{b=2}
		\correctchoice{stimme eher nicht zu}\scoring{b=3}
		\correctchoice{stimme gar nicht zu}\scoring{b=4}
		\vspace{-0.05cm}
		\\ . . . . . . . . . . . 
		\vspace{-0.05cm}
		\correctchoice{keine Angabe}\scoring{b=88}
	\end{choices}
\end{questionmult} 


\pagebreak


\fr{Wir leben zwar in Nationalstaaten, aber die Menschen sind viel zu verschieden, als dass es noch eine Rolle spielen sollte.}
\vspace{-0.2cm}
\begin{questionmult}{latent05}\scoring{v=-1}
	\begin{choices}[o]
		\correctchoice{stimme voll zu}\scoring{b=1} 
		\correctchoice{stimme eher zu}\scoring{b=2}
		\correctchoice{stimme eher nicht zu}\scoring{b=3}
		\correctchoice{stimme gar nicht zu}\scoring{b=4}
		\vspace{-0.05cm}
		\\ . . . . . . . . . . . 
		\vspace{-0.05cm}
		\correctchoice{keine Angabe}\scoring{b=88}
	\end{choices}
\end{questionmult} 

\vspace{0.3cm}

\fr{Ereignisse wie die Flüchtlingskrise sind Resultat eines gezielten Plans, der nicht von allen durchschaut wird.}
\vspace{-0.2cm}
\begin{questionmult}{latent06}\scoring{v=-1}
	\begin{choices}[o]
		\correctchoice{stimme voll zu}\scoring{b=1} 
		\correctchoice{stimme eher zu}\scoring{b=2}
		\correctchoice{stimme eher nicht zu}\scoring{b=3}
		\correctchoice{stimme gar nicht zu}\scoring{b=4}
		\vspace{-0.05cm}
		\\ . . . . . . . . . . . 
		\vspace{-0.05cm}
		\correctchoice{keine Angabe}\scoring{b=88}
	\end{choices}
\end{questionmult} 

\vspace{0.3cm}

\fr{In Anbetracht der vielen Konflikte auf der Welt müssen wir endlich aufwachen und erkennen wo die eigentlichen Schuldigen sitzen.}
\vspace{-0.2cm}
\begin{questionmult}{latent07}\scoring{v=-1}
	\begin{choices}[o]
		\correctchoice{stimme voll zu}\scoring{b=1} 
		\correctchoice{stimme eher zu}\scoring{b=2}
		\correctchoice{stimme eher nicht zu}\scoring{b=3}
		\correctchoice{stimme gar nicht zu}\scoring{b=4}
		\vspace{-0.05cm}
		\\ . . . . . . . . . . . 
		\vspace{-0.05cm}
		\correctchoice{keine Angabe}\scoring{b=88}
	\end{choices}
\end{questionmult} 

\vspace{0.3cm}

\fr{Wir sollten nicht künstlich in die natürliche Ordnung der nationalen Gemeinschaften eingreifen.}
\vspace{-0.2cm}
\begin{questionmult}{latent08}\scoring{v=-1}
	\begin{choices}[o]
		\correctchoice{stimme voll zu}\scoring{b=1} 
		\correctchoice{stimme eher zu}\scoring{b=2}
		\correctchoice{stimme eher nicht zu}\scoring{b=3}
		\correctchoice{stimme gar nicht zu}\scoring{b=4}
		\vspace{-0.05cm}
		\\ . . . . . . . . . . . 
		\vspace{-0.05cm}
		\correctchoice{keine Angabe}\scoring{b=88}
	\end{choices}
\end{questionmult} 

\pagebreak

\fr{Die Gier einer hemmungslosen Finanzelite ist das Grundproblem unserer Gesellschaft.}
\vspace{-0.2cm}
\begin{questionmult}{latent09}\scoring{v=-1}
	\begin{choices}[o]
		\correctchoice{stimme voll zu}\scoring{b=1} 
		\correctchoice{stimme eher zu}\scoring{b=2}
		\correctchoice{stimme eher nicht zu}\scoring{b=3}
		\correctchoice{stimme gar nicht zu}\scoring{b=4}
		\vspace{-0.05cm}
		\\ . . . . . . . . . . . 
		\vspace{-0.05cm}
		\correctchoice{keine Angabe}\scoring{b=88}
	\end{choices}
\end{questionmult} 

\vspace{0.3cm}

\fr{Wenn es uns nicht gelingt, die im Verborgenen agierende Weltregierung zu beseitigen, wird sie die Welt in den Abgrund stürzen.}
\vspace{-0.2cm}
\begin{questionmult}{latent10}\scoring{v=-1}
	\begin{choices}[o]
		\correctchoice{stimme voll zu}\scoring{b=1} 
		\correctchoice{stimme eher zu}\scoring{b=2}
		\correctchoice{stimme eher nicht zu}\scoring{b=3}
		\correctchoice{stimme gar nicht zu}\scoring{b=4}
		\vspace{-0.05cm}
		\\ . . . . . . . . . . . 
		\vspace{-0.05cm}
		\correctchoice{keine Angabe}\scoring{b=88}
	\end{choices}
\end{questionmult} 

\vspace{0.3cm}

\fr{Unsere kapitalistische Gesellschaft zeichnet sich dadurch aus, dass alle Menschen strukturellen Zwängen unterliegen und niemand die volle Kontrolle darüber hat.}
\vspace{-0.2cm}
\begin{questionmult}{latent11}\scoring{v=-1}
	\begin{choices}[o]
		\correctchoice{stimme voll zu}\scoring{b=1} 
		\correctchoice{stimme eher zu}\scoring{b=2}
		\correctchoice{stimme eher nicht zu}\scoring{b=3}
		\correctchoice{stimme gar nicht zu}\scoring{b=4}
		\vspace{-0.05cm}
		\\ . . . . . . . . . . . 
		\vspace{-0.05cm}
		\correctchoice{keine Angabe}\scoring{b=88}
	\end{choices}
\end{questionmult} 

\vspace{0.3cm}

\fr{Die, die uns regieren, handeln nicht im Interesse des Volkes.}
\vspace{-0.2cm}
\begin{questionmult}{latent12}\scoring{v=-1}
	\begin{choices}[o]
		\correctchoice{stimme voll zu}\scoring{b=1} 
		\correctchoice{stimme eher zu}\scoring{b=2}
		\correctchoice{stimme eher nicht zu}\scoring{b=3}
		\correctchoice{stimme gar nicht zu}\scoring{b=4}
		\vspace{-0.05cm}
		\\ . . . . . . . . . . . 
		\vspace{-0.05cm}
		\correctchoice{keine Angabe}\scoring{b=88}
	\end{choices}
\end{questionmult} 


\pagebreak


\section*{Beziehung zum Judentum}
\fr{Im Folgenden sehen Sie Aussagen über Jüdinnen und Juden wie sie immer wieder geäußert werden. Bitte geben Sie an, inwieweit Sie diesen zustimmen.} \\

\vspace{-0.5cm}

\tikzmark{f4a} % Auskommentieren, falls keine Trennlinie 

\headerfive{5.65cm}{stimme voll zu}{stimme eher zu}{stimme eher nicht zu}{stimme gar nicht zu}{keine Angabe} \\

\begin{colored}{gray!25} % Auskommentieren, falls keine Colorierung der Zeilen
	\itemsfive{mAS1}{6cm}{Auch heute noch ist der Einfluss der Juden zu groß.}{1}{2}{3}{4}{88} \\  	
	\itemsfive{mAS2}{6cm}{Die Juden arbeiten mehr als andere Menschen mit üblen Tricks, um das zu erreichen, was sie wollen.}{1}{2}{3}{4}{88} \\ 	
	\itemsfive{mAS3}{6cm}{Die Juden haben einfach etwas Besonderes und Eigentümliches an sich und passen nicht so recht zu uns.}{1}{2}{3}{4}{88} \\ 	
	\itemsfive{mAS4}{6cm}{Juden haben in Deutschland zu viel Einfluss.}{1}{2}{3}{4}{88} \\ 
	\itemsfive{mAS5}{6cm}{Durch ihr Verhalten sind Juden an ihren Verfolgungen mitschuldig.}{1}{2}{3}{4}{88} \\
	\itemsfive{lAS1}{6cm}{Ich glaube, dass sich viele nicht trauen, ihre wirkliche Meinung über Juden zu sagen.}{1}{2}{3}{4}{88} \\  
	\itemsfive{lAS2}{6cm}{Mir ist das ganze Thema "`Juden"' irgendwie unangenehm.}{1}{2}{3}{4}{88} \\  
	\itemsfive{lAS3}{6cm}{Was ich über Juden denke, sage ich nicht jedem.}{1}{2}{3}{4}{88} \\ 
\end{colored} % Auskommentieren, falls keine Colorierung der Zeilen

\tikzmark{f4b} % Auskommentieren, falls keine Trennlinie 

\gestrichelt{12.6}{12.6}{f4a}{f4b} % manuell positionieren
%\durchgezogen{12.6}{12.6}{f4a}{f4b} % manuell positionieren

\vspace{3cm}

Vielen Dank für Ihre Mitarbeit.


\clearpage\thispagestyle{empty}\null\newpage


%\def\AMCntextGoto{\par{\bf\emph{Please code the answer on
%the separate answer sheet.}}}
%  \AMCassociation{\id}
}


%\csvreader[head to column names]{\ExamInstanceDir/students.csv}{}{\klausur}

\end{document}
%---------------------------------------------
% Ende




